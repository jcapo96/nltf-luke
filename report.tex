\documentclass[12pt]{article}
\usepackage[a4paper, margin=2.5cm]{geometry}
\usepackage{caption}
\usepackage{float}
\usepackage{graphicx}
\usepackage{amsmath}
\usepackage{bookmark}
\usepackage{hyperref}
\usepackage{fancyhdr}
\usepackage{url}
\usepackage{graphicx}
\usepackage{cleveref}
\usepackage{makecell}
\usepackage{lipsum} % For placeholder text
\usepackage[font=small,labelfont=bf]{caption}
\usepackage{sectsty}
\sectionfont{\large}
\subsectionfont{\normalsize}
\fancypagestyle{firstpage}{
    \fancyhf{}
    \fancyhead[L]{
        \includegraphics[height=0.75cm]{fermilab_logo.png}
    }
    \fancyhead[C]{}
    \fancyhead[R]{
        Author: Jordi Capó (\href{mailto:jordi.capo@ific.uv.es}jordi.capo@ific.uv.es)\\
        Approved by: Flor de Maria Blaszczyk (\href{mailto:fblaszcz@fnal.gov}fblaszcz@fnal.gov)
    }
}
\thispagestyle{firstpage}

\begin{document}

\begin{center}
    \LARGE \textbf{ MSU AL Coated Cathode } \\
    \large Date of Receipt: 06/27/2025
\end{center}

\begin{center}
    \large Material Tests Stand: \textbf{Luke} - \textit{Noble Liquid Test Facility (NLTF), FNAL} \\
    \small Contamination, outgassing and cryogenic compatibility studies for Argon detectors.
\end{center}

% \vspace{1cm}

\begin{table}[H]
    \centering
    \caption*{\textbf{Sample Information}}
    \resizebox{\textwidth}{!}{
    \begin{tabular}{|l|p{0.75\textwidth}|}
        \hline
        \textbf{Name} & MSU AL Coated Cathode \\ \hline
        \textbf{Composition} &  \\ \hline
        \textbf{Pictures} & \hyperref[sec:pictures]{Sample Pictures~\ref{sec:pictures}} \\ \hline
        \textbf{Dimensions} & 10 equal plates of 10 by 10 by 1 cm each: 2400 cm² \\ \hline
        \textbf{Source} & Old samples from the NLTF \\ \hline
        \textbf{Preparation} & Ten stainless steel plates were prepared for testing. Each sample was mechanically cleaned via cutting and abrasion with sandpaper to remove surface contaminants and markings, followed by degreasing with 200 proof ethanol. Subsequently, samples were immersed for 24 hours in a 65 Celsius ultrasonic bath containing [Micro90 ColePalmer] to eliminate residual contaminants. Post-bath, they were rinsed with distilled water to remove solvent traces and then re-cleaned with ethanol 200 proof. To minimize adsorbed moisture, all samples were baked at 110 Celsius for 1 hour in a convection oven and left to cool overnight inside the oven at room temperature until testing. For ease of handling, samples were grouped into two sets of five, connected through corner perforations using stainless steel wire commonly employed in material compatibility testing. \\ \hline
    \end{tabular}
    }
\end{table}

Three consecutive runs were performed: baseline (no sample), ullage (sample in gas phase), and liquid (sample submerged in liquid argon). Each run began when the filter was deactivated after reaching a plateau in electron lifetime and H\textsubscript{2}O concentration. Runs last typically 24hrs, making sure a plateau in the water outgassing is observed. Details are summarized in Table~\ref{tab:measurement_summary}.

\begin{table}[H]
    \centering
    \caption*{\textbf{Measurement Summary}}
    \renewcommand\arraystretch{1.5} % Increase row height
    \resizebox{\textwidth}{!}{
    \begin{tabular}{|l|c|c|c|c|c|c|}
        \hline
        \makecell{\rule{0pt}{2.5ex}\textbf{Run}} &
        \makecell{\rule{0pt}{2.5ex}\textbf{Start Date}} &
        \makecell{\rule{0pt}{2.5ex}\textbf{End Date}} &
        \makecell{\rule{0pt}{2.5ex}\textbf{Initial H\textsubscript{2}O} \\ \textbf{[ppb]}} &
        \makecell{\rule{0pt}{2.5ex}\textbf{Final H\textsubscript{2}O} \\ \textbf{[ppb]}} &
        \makecell{\rule{0pt}{2.5ex}\textbf{$\Delta$ H\textsubscript{2}O} \\ \textbf{[ppb]}} &
        \makecell{\rule{0pt}{2.5ex}\textbf{Temp.} \\ \textbf{[K]}} \\
        \hline
        Baseline & 11/20/22 01:44 & 11/20/22 22:46 & 0.8 $\pm$ 0.15 & 1.0 $\pm$ 0.2 & 0.2 $\pm$ 0.2 & -- \\
        \hline
        Ullage & 12/15/22 21:12 & 12/16/22 14:00 & 0.62 $\pm$ 0.14 & 0.7 $\pm$ 0.2 & 0.1 $\pm$ 0.2 & 151.9 $\pm$ 0.4\\
        \hline
        Liquid & 12/08/22 18:11 & 12/09/22 18:11 & 0.39 $\pm$ 0.07 & 73.6 $\pm$ 31.7 & 73.2 $\pm$ 31.7 & 94.5 $\pm$ 0.0\\
        \hline
    \end{tabular}
    }
    \renewcommand\arraystretch{1} % Reset row height
    \vspace{0.3cm}
    \centering
    \caption{Summary of measurement results for each run. Systematic uncertainties are not included.}
    \vspace{0.1cm}
    \label{tab:measurement_summary}
\end{table}

\vspace{-1cm}

\section*{Conclusions on the Analysis}
The experimental data indicate a distinct behavior when the samples are immersed in liquid argon. Specifically, the equilibration time for water vapor was significantly longer compared to both baseline and ullage conditions. This observation suggests that water outgassing from the stainless steel samples is inhibited when submerged, likely due to the reduced mobility of water molecules in the liquid phase of argon. The suppression of water diffusion in liquid argon is consistent with established literature. Throughout all experimental runs, no measurable degradation in electron lifetime was detected.

\section*{Summary of Plots}

All plots show the data taken after the filter is turned off. The initial water concentration is averaged over 60 minutes before the filter is turned off; the final water concentration is averaged over 8 hours once the plateau is reached.

\begin{figure}[H]
    \centering
    \includegraphics[width=0.9\textwidth]{ purity.png }
    \centering
    \caption{Electron lifetime decay over time for the baseline, ullage and liquid tests respectively. Filter is turned-off at $t=0 s$.}
    \label{fig:purity}
\end{figure}

\begin{figure}[H]
    \centering
    \includegraphics[width=0.9\textwidth]{ h2o_concentration.png }
    \centering
    \caption{H\textsubscript{2}O Concentration evolution for the baseline, ullage and liquid runs. Filter is turned-off at $t=0 s$.}
    \label{fig:h2o}
\end{figure}

\begin{figure}[H]
    \centering
    \includegraphics[width=0.9\textwidth]{ temperature.png }
    \centering
    \caption{Temperature evolution for the ullage and liquid runs, once the filter is turned off. The temperature is measured on the platform holding the sample.}
    \label{fig:temperature}
\end{figure}

\begin{figure}[H]
    \centering
    \includegraphics[width=0.9\textwidth]{ level.png }
    \centering
    \caption{Liquid argon level in the cryostat for each run. Filter is turned-off at $t=0 s$. The level drops over time because some gas is lost to measure the water level to the gas analyzer.}
    \label{fig:level}
\end{figure}

\section*{Sample Pictures}
\label{sec:pictures}

\begin{figure}[H]
    \centering
    \begin{minipage}{\textwidth}
        \centering
        \includegraphics[width=\textwidth,height=0.4\textheight,keepaspectratio]{ /Users/jcapo/cernbox/NLTFdata/STAINLESS_STEEL/SAMPLE_PREPARATION/stainless_steel_10_before1.jpeg }
        \centering
        \caption{Sample before tests.}
        \label{fig:sample_before}
    \end{minipage}

    \vspace{0.02\textheight}

    \begin{minipage}{\textwidth}
        \centering
        \includegraphics[width=\textwidth,height=0.4\textheight,keepaspectratio]{ /Users/jcapo/cernbox/NLTFdata/STAINLESS_STEEL/SAMPLE_PREPARATION/stainless_steel_10_before1.jpeg }
        \centering
        \caption{Sample after tests.}
        \label{fig:sample_after}
    \end{minipage}
\end{figure}

\end{document}