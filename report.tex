\documentclass[12pt]{article}
\usepackage[a4paper, margin=2.5cm]{geometry}
\usepackage{caption}
\usepackage{float}
\usepackage{graphicx}
\usepackage{amsmath}
\usepackage{bookmark}
\usepackage{hyperref}
\usepackage{fancyhdr}
\setlength{\headheight}{28pt}
\usepackage{url}
\usepackage{graphicx}
\usepackage{cleveref}
\usepackage{makecell}
\usepackage{lipsum} % For placeholder text
\usepackage[font=small,labelfont=bf]{caption}
\usepackage{sectsty}
\sectionfont{\large}
\subsectionfont{\normalsize}
\fancypagestyle{firstpage}{
    \fancyhf{}
    \fancyhead[L]{
        \includegraphics[height=0.75cm]{fermilab_logo.png}
    }
    \fancyhead[C]{}
    \fancyhead[R]{
        Author: Jordi Capó (\href{mailto:jordi.capo@ific.uv.es}jordi.capo@ific.uv.es)\\
        Approved by: Flor de Maria Blaszczyk (\href{mailto:fblaszcz@fnal.gov}fblaszcz@fnal.gov)
    }
}
\thispagestyle{firstpage}

\begin{document}

\begin{center}
    \LARGE \textbf{ ABS (Thermoplastic polymer) } \\
    \large Date of Receipt: 03/06/2025
\end{center}

\begin{center}
    \large Material Tests Stand: \textbf{Luke} - \textit{Noble Liquid Test Facility (NLTF), FNAL} \\
    \small Contamination, outgassing and cryogenic compatibility studies for Argon detectors.
\end{center}

% \vspace{1cm}

\begin{table}[H]
    \centering
    \caption*{\textbf{Sample Information}}
    \resizebox{\textwidth}{!}{
    \begin{tabular}{|l|p{0.75\textwidth}|}
        \hline
        \textbf{Name} & ABS (Thermoplastic polymer) \\ \hline
        \textbf{Composition} & Acrylonitrile Butadiene Styrene \\ \hline
        \textbf{Pictures} & \hyperref[sec:pictures]{Sample Pictures~\ref{sec:pictures}} \\ \hline
        \textbf{Dimensions} & 2 samples of 2.5 in. x 2.5 in. x 1/4 in. \\ \hline
        \textbf{Source} & Jon Urheim (Indiana University) \\ \hline
        \textbf{Preparation} & Cleaned with Ethyl alcohol 200 proof \\ \hline
    \end{tabular}
    }
\end{table}

Three consecutive runs were performed: baseline (no sample), ullage (sample in gas phase), and liquid (sample submerged in liquid argon). Each run began when the filter was deactivated after reaching a plateau in electron lifetime and H\textsubscript{2}O concentration. Runs last typically 24hrs, making sure a plateau in the water outgassing is observed. Details are summarized in Table~\ref{tab:measurement_summary}.

\begin{table}[H]
    \centering
    \caption*{\textbf{Measurement Summary}}
    \renewcommand\arraystretch{1.5} % Increase row height
    \resizebox{\textwidth}{!}{
    \begin{tabular}{|l|c|c|c|c|c|c|}
        \hline
        \makecell{\rule{0pt}{2.5ex}\textbf{Run}} &
        \makecell{\rule{0pt}{2.5ex}\textbf{Start Date}} &
        \makecell{\rule{0pt}{2.5ex}\textbf{End Date}} &
        \makecell{\rule{0pt}{2.5ex}\textbf{Initial H\textsubscript{2}O} \\ \textbf{[ppb]}} &
        \makecell{\rule{0pt}{2.5ex}\textbf{Final H\textsubscript{2}O} \\ \textbf{[ppb]}} &
        \makecell{\rule{0pt}{2.5ex}\textbf{$\Delta$ H\textsubscript{2}O} \\ \textbf{[ppb]}} &
        \makecell{\rule{0pt}{2.5ex}\textbf{Temp.} \\ \textbf{[K]}} \\
        \hline
        Baseline & 03/25/25 10:08 & 03/26/25 09:00 & 1.65 $\pm$ 0.05 & 5.7 $\pm$ 0.1 & 4.1 $\pm$ 0.1 & -- \\
        \hline
        Ullage & 03/28/25 14:21 & 03/29/25 13:20 & 4.23 $\pm$ 0.07 & 16.8 $\pm$ 0.7 & 12.6 $\pm$ 0.7 & 217.1 $\pm$ 0.2\\
        \hline
        Liquid & 04/01/25 15:23 & 04/02/25 14:20 & 0.99 $\pm$ 0.09 & 0.3 $\pm$ 0.1 & -0.7 $\pm$ 0.1 & 94.4 $\pm$ 0.1\\
        \hline
    \end{tabular}
    }
    \renewcommand\arraystretch{1} % Reset row height
    \vspace{0.3cm}
    \centering
    \caption{Summary of measurement results for each run. Systematic uncertainties are not included.}
    \vspace{0.1cm}
    \label{tab:measurement_summary}
\end{table}

\vspace{-1cm}

\section*{Conclusions on the Analysis}
Water outgassing occurs when the sample is in the ullage. No visible impact on electron lifetime. No visible physical damage on the sample.

\section*{Summary of Plots}

All plots show the data taken after the filter is turned off. The initial water concentration is averaged over 60 minutes before the filter is turned off; the final water concentration is averaged over 480 hours once the plateau is reached.

\begin{figure}[H]
    \centering
    \includegraphics[width=0.9\textwidth]{ purity.png }
    \centering
    \caption{Electron lifetime decay over time for the baseline, ullage and liquid tests respectively. Filter is turned-off at $t=0 s$.}
    \label{fig:purity}
\end{figure}

\begin{figure}[H]
    \centering
    \includegraphics[width=0.9\textwidth]{ h2o_concentration.png }
    \centering
    \caption{H\textsubscript{2}O Concentration evolution for the baseline, ullage and liquid runs. Filter is turned-off at $t=0 s$.}
    \label{fig:h2o}
\end{figure}

\begin{figure}[H]
    \centering
    \includegraphics[width=0.9\textwidth]{ temperature.png }
    \centering
    \caption{Temperature evolution for the ullage and liquid runs, once the filter is turned off. The temperature is measured on the platform holding the sample.}
    \label{fig:temperature}
\end{figure}

\begin{figure}[H]
    \centering
    \includegraphics[width=0.9\textwidth]{ level.png }
    \centering
    \caption{Liquid argon level in the cryostat for each run. Filter is turned-off at $t=0 s$. The level drops over time because some gas is lost to measure the water level to the gas analyzer.}
    \label{fig:level}
\end{figure}

\section*{Sample Pictures}
\label{sec:pictures}

\begin{figure}[H]
    \centering
    \begin{minipage}{\textwidth}
        \centering
        \includegraphics[width=\textwidth,height=0.4\textheight,keepaspectratio]{ /Users/jcapo/cernbox/NLTFdata/ABS/IMAGES/ABS_before1.jpg }
        \centering
        \caption{Sample before tests.}
        \label{fig:sample_before}
    \end{minipage}

    \vspace{0.02\textheight}

    \begin{minipage}{\textwidth}
        \centering
        \includegraphics[width=\textwidth,height=0.4\textheight,keepaspectratio]{ /Users/jcapo/cernbox/NLTFdata/ABS/IMAGES/ABS_after1.jpg }
        \centering
        \caption{Sample after tests.}
        \label{fig:sample_after}
    \end{minipage}
\end{figure}

\end{document}